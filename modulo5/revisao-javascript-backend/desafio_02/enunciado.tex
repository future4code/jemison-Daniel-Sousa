## Cadastro de cliente

Crie um programa que insere um novo cliente no array de clientes.

### Lista de clientes já cadastrados

```jsx
[
	{ id: 1, nome: “Fulano” },
	{ id: 2, nome: “Ciclano” },
	{ id: 3, nome: “Beltrano” },
	{ id: 4, nome: “Fulana” }
]
```

### Entrada

Um objeto de mesmo formato que os existentes na lista de clientes.

### Saída

A lista atualizada de clientes.

### Validações

Depois de finalizar a implementação básica, trate os valores de entrada. Um cliente deve sempre possuir uma id numérica única e um nome em string sem caracteres especiais.

### Exemplos de implementação básica

cadastraCliente({ id: 5, nome: “Fulano” }) retorna a lista com esse novo cliente

### Exemplos de validações

cadastraCliente({ id: 1, nome: “Ciclana” }) retorna “Erro. Parâmetro inválido (’id já existe’).”

## Geração de tabuada

Crie um gerador de tabuada.

### Entrada

Um número de 1 a 10.

### Saída

Uma lista onde cada item representa uma operação da tabuada.

### Validações

Depois de finalizar a implementação básica, trate os valores de entrada.

O argumento deve ser um número entre 1 a 10.

### Exemplos de implementação básica

geraTabuada(5) retorna:

```jsx
[
	"5 x 0 = 0",
	"5 x 1 = 5",
	"5 x 2 = 10",
	"5 x 3 = 15",
	"5 x 4 = 20",
	... // e assim por diante até x 10
]
```

### Exemplos de validações

geraTabuada(”10”) retorna “Erro. Parâmetro inválido (’deve ser um número’).”

geraTabuada(50) retorna “Erro. Parâmetro inválido (’número precisa valer entre 1 e 10’).”